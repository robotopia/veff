\documentclass{article}

\usepackage{fullpage}
\usepackage{amsmath}

\title{Documentation for the \texttt{veff} package}
\author{Sam McSweeney}

\newcommand{\veff}{\texttt{veff}}
\newcommand{\parabfit}{\texttt{parabfit}}
\newcommand{\ggb}{\texttt{ggb}}

\newcommand{\x}{\texttt{x}}
\newcommand{\y}{\texttt{y}}
\newcommand{\s}{\texttt{s}}

\begin{document}

\maketitle

\section{Overview}

\veff{} is a set of programs designed for the analysis of pulsar secondary spectra.
It comprises three components: \parabfit{}, \veff{}, and \ggb{}.
\begin{itemize}
    \item \underline{\parabfit}: Fits a parabola to a secondary spectrum of the pulsar.
    \item \underline{\veff}: Analyses the relative motions of the Earth and the pulsar, projected onto the plane of the sky.
    \item \underline{\ggb}: Creates a visualisation of the Earth/pulsar system using the program Geogebra.
\end{itemize}

\section{Installation}

\section{The \parabfit{} program}

\parabfit{} takes as input a secondary spectrum of a pulsar and outputs the curvature of a fitted parabola.
The parabola fit is obtained via a generalised Hough Transform, with user-specified search parameters.

\subsection{Secondary spectrum format}

The secondary spectrum must be a text file containing three columns of numbers, here labeled \x{}, \y{}, and \s{}, respectively representing the $x$-coordinate, the $y$-coordinate, and the value at those coordinates of the secondary spectrum.
\x{} and \y{} \emph{must} be integers with the bottom-left pixel of the secondary spectrum corresponding to (\x{},\y{}) = (1,1).
The order of the lines do not matter, except that the last line must contain the top-right pixel.
(This is because the last line is used to discover how large to make the 2D array.)

\subsection{Command line arguments}

The basic usage of \parabfit{} is

\end{document}
